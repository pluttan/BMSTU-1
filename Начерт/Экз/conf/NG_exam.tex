
    \section*{\textbf{Инженерная графика}}\addcontentsline{toc}{section}{\textbf{Инженерная графика}}
\subsection*{1. \textit{Как расшифровывается аббревиатура ЕСКД?}}\addcontentsline{toc}{subsection}{1. \textit{Как расшифровывается аббревиатура ЕСКД?}}

ЕСКД - единая система конструкторской документации - комплекс государственных стандартов, устанавливающих правила, требования и нормы по разработке, оформлению и обращению конструкторской документации, разрабатываемой и применяемой на всех стадиях жизненного цикла изделия.
\subsection*{2. \textit{Обозначение основных форматов. Каково отношение сторон основных форматов?}}\addcontentsline{toc}{subsection}{2. \textit{Обозначение основных форматов. Каково отношение сторон основных форматов?}}
\begin{center}
\begin{tabular}{|c|c|c|c|c|c|c|c|}
\hline




Обозначение формата&A0&A1&A2&A3&A4&A5&\\ \hline

Размеры сторон&$841\times 1189$&$594\times 841$&$420\times 594$&$297\times 420$&$210\times 297$&$148\times 210$&\\ \hline

\hline
\end{tabular}
\end{center}
\subsection*{3. \textit{Что называют масштабом изображения?}}\addcontentsline{toc}{subsection}{3. \textit{Что называют масштабом изображения?}}

\textbf{Масштабом} называется отношение линейных размеров изображения детали к действительным разерам изображаемой детали.
\subsection*{4. \textit{Ряд масштабов уменьшения и увеличения.}}\addcontentsline{toc}{subsection}{4. \textit{Ряд масштабов уменьшения и увеличения.}}
\begin{center}
\begin{tabular}{|c|c|c|c|c|c|}
\hline




Масштабы уменьшения&$1:2$&$1:2,5$&$1:4$&$1:5$&$1:10$\\ \hline

Масштаб натуральной величины&$1:1$&&&&\\ \hline

Масштабы увеличения&$2:1$&$2,5:1$&$4:1$&$5:1$&$10:1$\\ \hline

\hline
\end{tabular}
\end{center}
\subsection*{5. \textit{Как указывают масштаб на чертеже?}}\addcontentsline{toc}{subsection}{5. \textit{Как указывают масштаб на чертеже?}}

Масштаб указывается в предназначенной для этого графе основной надписи чертежа. Обозначается по типу $1:1$; $1:2$; $2:1$ и т.д.
\subsection*{6. \textit{Назначение основной сплошной толстой линии, сплошной тонкой линии, штрихпунктирной линии, штриховой линии. Указать параметры этих линий.}}\addcontentsline{toc}{subsection}{6. \textit{Назначение основной сплошной толстой линии, сплошной тонкой линии, штрихпунктирной линии, штриховой линии. Указать параметры этих линий.}}

\textbf{Основная сплошная толстая линия} - применяется для изображения видимого контура предмета, линий пересечения поверхностей и контура сечения (вынесенного или входящего в состав разреза).

\textbf{Сплошная тонкая линия} - применяется для изображения линий построения, выносных и размерных линий, а также для: линий контура наложенного сечения, линий-выносок, шриховки сечений.

\textbf{Штрихпунктирная линия} - применяется для изображения осевых и центровых линий.

\textbf{Штриховая линия} - применяется для изображения линий невидимого контура.


\image{img/6.jpg}{150}

\vspace{14pt} 
\begin{center}
\begin{tabular}{|c|c|}
\hline




Основная сплошная толстая линия&$s$\\ \hline

Сплошная тонкая линия&От $s/3$ до $s/2$\\ \hline

Штрихпунктирная линия&От $s/2$ до $2s/3$\\ \hline

Штриховая линия&От $s/3$ до $s/2$\\ \hline

\hline
\end{tabular}
\end{center}
% (0\_0) 
\subsection*{7. \textit{Ряд размеров шрифта. Каким размером букв определяется размер шрифта?}}\addcontentsline{toc}{subsection}{7. \textit{Ряд размеров шрифта. Каким размером букв определяется размер шрифта?}}

Устанавливаются следующие размеры шрифта:
\begin{center}
\begin{tabular}{|c|c|c|c|c|c|c|c|c|c|c|}
\hline




$1,8$мм&$2,5$мм&$3,5$мм&$5$мм&$7$мм&$10$мм&$14$мм&$20$мм&$28$мм&$40$мм\\ \hline

\hline
\end{tabular}
\end{center}

Размер шрифта определятся высотой прописных (заглавных) букв в мм.
\subsection*{8. \textit{Какое изображение называется видом?}}\addcontentsline{toc}{subsection}{8. \textit{Какое изображение называется видом?}}

\textbf{Видом} называется изображение, на котором показана обращенная к наблюдателю видимая часть поверхности предмета.
\subsection*{9. \textit{Как называются виды, получаемые на основных плоскостях проекций? Как располагают на чертеже основные виды?}}\addcontentsline{toc}{subsection}{9. \textit{Как называются виды, получаемые на основных плоскостях проекций? Как располагают на чертеже основные виды?}}


\image{img/9.jpg}{260}

\subsection*{10. \textit{Какое изображене предмета на чертеже принимают в качестве главного? Какие к нему требования?}}\addcontentsline{toc}{subsection}{10. \textit{Какое изображене предмета на чертеже принимают в качестве главного? Какие к нему требования?}}

Главный вид должен давать наиболее полное представление о форме и размерах детали.
\subsection*{11. \textit{Какое изображение называют дополнительным видом, местным видом?}}\addcontentsline{toc}{subsection}{11. \textit{Какое изображение называют дополнительным видом, местным видом?}}

\textbf{Дополнительный вид} - получается проецированием предмета на плоскость, не параллельную ни одной из основных плоскостей проекций. Применяется в тех случаях, когда изображение предмета или его элемента не может быть показано на основных линиях без искажения формы и размеров.

\textbf{Местный вид} - изображение отдельного, ограниченного места поверхности предмета. Применяется в тех случаях, когда из всего вида только часть его необходима для уточнения формы предмета.
\subsection*{12. \textit{Какое изображение называется разрезом?}}\addcontentsline{toc}{subsection}{12. \textit{Какое изображение называется разрезом?}}

\textbf{Разрезом} называется изображение предмета, полученное при мысленном рассечении его одной или несколькими секущими плоскостями. При этом часть предмета, расположенная между наблюдателем и секущей плоскостью, мысленно удаляется, а на плоскости проекций изображается то, что получается в секущей плоскости и что расположено за ней.
\subsection*{13. \textit{Как обозначают разрезы на чертежах в общем случае?}}\addcontentsline{toc}{subsection}{13. \textit{Как обозначают разрезы на чертежах в общем случае?}}

Положение секущей плоскости указывается разомкнутой линией. Штрихи разомкнутой линии не должны пересекать контур детали. На штрихах разомкнутой линии перпендикулярно к ним ставят стрелки, указывающие направления взгляда. Около каждой стрелки наносится одна и та же прописная буква.


\image{img/13\_1.jpg}{210}


Надпись над разрезом содержит две буквы, которыми обозначена секущая плоскость, написанные через тире.

Фигура сечения предмета заштриховывается тонкими линиями под углом $45^{\circ}$ (если при этом линии штриховки параллельны линиям контура предмета или осевым линиям, то допускаются углы $30^{\circ}$ и $60^{\circ}$). Их наклон может выполняться влево или вправо, но в одну сторону на всех сечениях, относящихся к одной и той же детали.


\image{img/13\_2.jpg}{260}

\subsection*{14. \textit{В каких случаях не указывают положение секущей плоскости при выполнении разреза?}}\addcontentsline{toc}{subsection}{14. \textit{В каких случаях не указывают положение секущей плоскости при выполнении разреза?}}

В случаях, когда \textbf{секущая плоскость совпадает с плоскостью симметрии предмета}, не указывают положение секущей плоскости при выполнении разреза.
\subsection*{15. \textit{Как называются разрезы, расположенные на месте соответствующих видов?}}\addcontentsline{toc}{subsection}{15. \textit{Как называются разрезы, расположенные на месте соответствующих видов?}}

\textbf{Горизонтальные разрезы} (секущая плоскость параллельна горизонтальной плоскости проекций), \textbf{фронтальные разрезы} (секущая плоскость параллельна фронтальной плоскости проекций) и \textbf{профильные разрезы} (секущая плоскость параллельна профильной плоскости проекций) могут размещаться на месте соответствующих основных видов.
\subsection*{16. \textit{Как разделяются разрезы в зависимости от числа секущих плоскостей?}}\addcontentsline{toc}{subsection}{16. \textit{Как разделяются разрезы в зависимости от числа секущих плоскостей?}}

\textbf{Простые разрезы} - разрезы, образованные одной секущей плоскостью.


\image{img/c16\_1.jpg}{150}


\textbf{Сложные разрезы} - разрезы, образованные двумя и более секущими плоскостями.


\image{img/c16\_2.png}{180}

\subsection*{17. \textit{Какие линии являются разделяющими при соединении части вида и части соответствующего разреза?}}\addcontentsline{toc}{subsection}{17. \textit{Какие линии являются разделяющими при соединении части вида и части соответствующего разреза?}}

Для соединения части вида и части разреза используются:
\begin{itemize}

\item Штрихпунктирные (осевые)


\image{img/c17\_1.png}{200}

\item Сплошные волнистые - если с границей части вида и разреза совпадает линия контура.


\image{img/c17\_2.png}{200}


\end{itemize}
\subsection*{18. \textit{Как показывают на разрезе тонкие стенки типа ребер жесткости, если секущая плоскость направлена вдоль их длинной стороны?}}\addcontentsline{toc}{subsection}{18. \textit{Как показывают на разрезе тонкие стенки типа ребер жесткости, если секущая плоскость направлена вдоль их длинной стороны?}}

Тонкие стенки типа ребер жесткости показывают \textbf{незаштрихованными}, если секущая плоскость проходит вдоль длинной стороны элемента.


\image{img/c18.jpg}{150}

\subsection*{19. \textit{Какое изображение называют сечением? Какое сечение называют вынесенным, наложенным?}}\addcontentsline{toc}{subsection}{19. \textit{Какое изображение называют сечением? Какое сечение называют вынесенным, наложенным?}}

\textbf{Сечение} - ортогональная проекция фигуры, получающейся в одной или нескольких секущих плоскостях или поверхностях при мысленном рассечении проецируемого предмета. На сечении показывают только то, что находится непосредственно в секущей плоскости.

\textbf{Вынесенное сечение} - сечение, располагающееся на свободном поле чертежа.

\textbf{Наложенное сечение} - сечение, располагающееся непосредственно на изображении предмета.


\image{img/c19\_2.jpg}{250}

\subsection*{20. \textit{Какие сечения не обозначают на чертеже?}}\addcontentsline{toc}{subsection}{20. \textit{Какие сечения не обозначают на чертеже?}}

При выполнении \textbf{вынесенных} и \textbf{наложенных симметричных} сечений положение секущей плоскости не указывается.
\subsection*{21. \textit{В каких случаях сечение следует заменять разрезом?}}\addcontentsline{toc}{subsection}{21. \textit{В каких случаях сечение следует заменять разрезом?}}

Если сечение получается состоящим из отдельных частей, то сечение должно быть заменено разрезом.


\image{img/c21.png}{180}

\subsection*{22. \textit{Как графически на чертежах обозначают материалы в сечениях, на разрезах?}}\addcontentsline{toc}{subsection}{22. \textit{Как графически на чертежах обозначают материалы в сечениях, на разрезах?}}

Материалы в сечениях и разрезах обозначают с помощью разных типов штриховки.


\image{img/c22.jpg}{250}

\subsection*{23. \textit{Как выбирают направление линий штриховки и расстояние между ними для разных изображений одного и того же предмета на чертеже?}}\addcontentsline{toc}{subsection}{23. \textit{Как выбирают направление линий штриховки и расстояние между ними для разных изображений одного и того же предмета на чертеже?}}

Линии штриховки должны проводиться под углом $45^{\circ}$ (если при этом линии штриховки параллельны линиям контура предмета или осевым линиям, то допускаются углы $30^{\circ}$ и $60^{\circ}$). Их наклон может выполняться влево или вправо, но в одну сторону на всех сечениях, относящихся к одной и той же детали.

Расстояния между линиями штриховки должны быть одинаковыми для всех выполняемых в одном и том же масштабе сечений данной детали. Это расстояние выбирается от $1$ до $10 \text{мм}$, в зависимости от площади штриховки: большее расстояние соответствует большей площади фигуры сечения.
\subsection*{24. \textit{Чему равно минимальное растояние между размерной линией и линией контура изображения, между параллельными размерными линиями?}}\addcontentsline{toc}{subsection}{24. \textit{Чему равно минимальное растояние между размерной линией и линией контура изображения, между параллельными размерными линиями?}}

Минимальное расстояние между параллельными размерными линиями составляет $7$мм, а между размерной линией и линией контура - $10$мм.
\subsection*{25. \textit{В каких единицах измерения указывают размеры на чертежах?}}\addcontentsline{toc}{subsection}{25. \textit{В каких единицах измерения указывают размеры на чертежах?}}

Линейные размеры принято наносить в миллиметрах без указания единицы измерения. Если на чертеже нужно указать размеры не в мм, а в других единицах измерения, то соответствующие размерные числа записывают с обозначением единицы измерения.

    